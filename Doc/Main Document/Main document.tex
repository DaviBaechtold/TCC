% --------------------------------------------------------------------
% PACOTES E CONFIGURAÇÕES DO DOCUMENTO
% --------------------------------------------------------------------
\documentclass[
	a4paper,      % Tamanho do papel
	12pt,         % Tamanho da fonte
	english,      % Idioma principal (para hifenização, etc.)
	oneside,      % Evita layout frente/verso que pode inserir páginas em branco
	openany       % Não força capítulos a começarem em página ímpar (direita)
]{abntex2}

% Codificação de caracteres e fontes
\usepackage[T1]{fontenc}      % Usa fontes de 8-bits, melhora a hifenização
\usepackage[utf8]{inputenc}   % Permite usar acentos diretamente no código
\usepackage[brazil]{babel}
\usepackage{lmodern}          % Usa a fonte Latin Modern, uma versão melhorada do Computer Modern
\usepackage{indentfirst}      % Indenta o primeiro parágrafo de cada seção

% Pacotes para referências bibliográficas (padrão ABNT)
\usepackage[brazilian,hyperpageref]{backref} % Cria links da bibliografia para o texto
\usepackage[alf]{abntex2cite}  % Estilo de citação alfabético (Ex: [AUTOR, ANO])

% Informações do trabalho para capa e folha de rosto
% --------------------------------------------------------------------
\titulo{Temporal Transformer-Based Gesture Recognition from Monocular 2D Skeleton Sequences}
\autor{Davi Baechtold Campos}
\instituicao{%
  Pontifícia Universidade Católica do Paraná --- PUCPR
  \par
  Campus Curitiba
  \par
  Curso de Engenharia de Computação
}
\tipotrabalho{Trabalho de Conclusão de Curso (TCC)}
\orientador{Prof. Dr. Alceu de Souza Brito Junior}
\local{Curitiba}
\data{\the\year} % Pega o ano atual automaticamente

% Preâmbulo para a folha de rosto (o texto que descreve o trabalho)
\preambulo{Trabalho de Conclusão de Curso apresentado como requisito parcial para a obtenção do grau de Bacharel em Engenharia de Computação, pelo Curso de Engenharia de Computação da Pontifícia Universidade Católica do Paraná.}

% --------------------------------------------------------------------
% INÍCIO DO DOCUMENTO
% --------------------------------------------------------------------
\begin{document}

% Seleciona o idioma para o documento
\selectlanguage{brazil}

% --------------------------------------------------------------------
% ELEMENTOS PRÉ-TEXTUAIS
% --------------------------------------------------------------------

% Gera a capa com as informações definidas acima
\imprimircapa

% Gera a folha de rosto com as informações definidas acima
\imprimirfolhaderosto

% --- DEDICATÓRIA ---
% O ambiente 'dedicatoria' cria uma página separada.
% O \vspace*{\fill} empurra o texto para o fim da página.
\begin{dedicatoria}
	\vspace*{\fill} % Move o texto para a parte inferior da página
	\begin{flushright} % Alinha o texto à direita
		\textit{Aos meus pais, pelo amor, apoio e incentivo incondicional.\\
		Aos meus amigos, pela jornada compartilhada.}
	\end{flushright}
\end{dedicatoria}

% --- AGRADECIMENTOS ---
% O ambiente 'agradecimentos' cria uma seção formatada.
\begin{agradecimentos}
	Agradeço primeiramente 

	Ao meu orientador, Prof. Dr. Alceu de Souza Brito Junior, pela paciência, conhecimento compartilhado e pela orientação fundamental para a realização deste trabalho.
	
	Aos meus pais e minha família, por todo o suporte, amor e incentivo que foram a base para que eu chegasse até aqui.
	
	Aos meus amigos e colegas de curso, pelas longas noites de estudo, pela ajuda mútua e pelos momentos de descontração que tornaram a caminhada mais leve.
	
	A todos os professores do curso de Engenharia de Computação da PUCPR, pelos ensinamentos que moldaram minha formação profissional e pessoal.

\end{agradecimentos}

% --- RESUMO E ABSTRACT ---
% (Adicione aqui quando for escrever)
% \begin{resumo}
% \end{resumo}
% \begin{abstract}
% \end{abstract}

% --- SUMÁRIO ---
\tableofcontents* % O asterisco remove o Sumário do próprio Sumário

% --- LISTAS (FIGURAS, TABELAS) ---
% \listoffigures*
% \listoftables*

% --------------------------------------------------------------------
% ELEMENTOS TEXTUAIS (O CONTEÚDO DO SEU TCC)
% --------------------------------------------------------------------
\textual

\chapter{Introdução}
Esta é a introdução do seu trabalho. Aqui você irá contextualizar o tema, apresentar a problemática, os objetivos, a justificativa e a estrutura do documento.

O reconhecimento de gestos humanos é uma área de pesquisa ativa em visão computacional...

% --------------------------------------------------------------------
% ELEMENTOS PÓS-TEXTUAIS
% --------------------------------------------------------------------
\postextual

% --- REFERÊNCIAS ---
% O comando \bibliography aponta para o arquivo .bib com suas referências
\bibliography{referencias}

\end{document}
